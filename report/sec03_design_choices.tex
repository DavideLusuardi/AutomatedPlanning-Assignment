
\section{Design Choices}
\label{sec3}

In this section we discuss some design choices and some assumptions that have been made 
analyzing the scenario and the different subproblems: further assumptions are required 
in order to define better the proposed problem.
% during the implementation of the PDDL domain and problem files.

% We can start 

Each crate can have just one content.
Each person can initially have some crates or none.
Each person can need some emergency supplies of specific content, or can need nothing.
We assume that the need of person of a specific content can be satisfied through the delivering 
to it of just one crate having that content (e.g. if Mario needs food, just one crate containing
food is sufficient, no more).
For this reason, the problem file should be defined in a way that does not introduce any form of inconsistency.
For example, we could have that Mario already have a crate containing food but at the same time he needs food.

We assume that a crate already delivered is no more available (TODO), even if it is redundant.


Initially, we need to consider the main object types used in the proposed scenario.
We can simply spot the following types: robot, person, crate, location, content, 
and, from the Problem 2 on, the assignment introduces the concept of carrier.
Each type can be easily understood without further explanations. 
We can note... TODO

We can now start describing the predicates.
% TODO

\subsection{Problem 1}
In order to model the location of robotic agents, people and crates, appropriates predicates should be defined.
We could simply use just one predicate to express objects location, but we preferred to define a predicate for 
each type of objects to improve clarity. In this way, we have defined the predicates 
\texttt{(robot\_at ?r - robot ?l - location)}, \texttt{(person\_at ?p - person ?l - location)} and 
\texttt{(crate\_at ?c - crate ?l - location)}.



As we have said, people need specific emergency supplies. To model this fact, the predicate
\texttt{(need ?p - person ?co - content)} has been defined. The predicate 
\texttt{(have ?p - person ?co - content)} is used to define that a person has a crate with that content.
In order to know if a crate has already been delivered, the predicate \texttt{(available ?c - crate)}
is used: we assume that when a crate is delivered is no more available to be picked up or moved.
TODO: non assumere ciò sarebbe inutile


To model that a robot can be empty or can hold a crate, the predicates \texttt{(empty ?r - robot)} and
\texttt{(hold ?r - robot ?c - crate)} are respectively used.
We note that the predicate \texttt{empty} is not strictly required but it is useful to simplify the modeling
and increase efficiency of the planner.




We can now start describing the actions. Three different actions have been modeled: \texttt{pick\_up}, 
\texttt{move} and \texttt{deliver}.

The action \texttt{pick\_up}, as the name suggests, models the robot that picks up a crate:
the robot and the crate should be at the same location, the robot should be empty and the crate available.

The action \texttt{move} indicates the movement of the robot from one location to another.

The action \texttt{deliver} indicates the delivering of a crate to a person: the person should need emergency
supplies of the same type of the content of the crate, the robot holds the crate and it is at the same location
of the person.


From Problem 2 on, the introduction of the carrier, poses the need to change a bit the model 
introducing the \texttt{carrier} type and the \texttt{(carrier\_at ?ca - carrier ?l - location)} predicate.
Instead, the predicates \texttt{hold} and \texttt{empty} have been removed in favor of the predicate
\texttt{(load ?ca - carrier ?c - crate)} which indicates that the crate is loaded into the carrier.
In order to model the variable capacity of the carriers, the function \texttt{(capacity ?ca - carrier) - number}
has been defined: the function is used to check to remaining capacity of a carrier before load a crate.
The action \texttt{move\_carrier} has also been added and indicates the robot that moves a carrier.

In Problem 3, the actions have been transformed in durative actions, specifying appropriate durations.
A new predicate \texttt{(busy ?r - robot)} has been added in order to model which actions can be executed 
in parallel. 