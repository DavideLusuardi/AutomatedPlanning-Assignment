
\section{Design Choices}
\label{sec3}

In this section, we discuss some design choices and some assumptions that have been made 
analyzing the scenario and the different subproblems. These assumptions are required 
in order to better delineate the problem.
% during the implementation of the PDDL domain and problem files.

We can start saying that each crate can have just one specific content, each person can need none or 
more emergency supplies and can initially have some crates. This implies that each person could need 
more crates.
% Each person , can need some emergency supplies of specific content, or can need nothing.
We assume that a single crate with specific content is enough to satisfy the person need of that emergency 
supply, i.e. if a person needs some food, it is sufficient to deliver it just one crate containing food, no more. 
% We assume that the need of person of a specific content can be satisfied through the delivering 
% to it of just one crate having that content (e.g. if Mario needs food, just one crate containing
% food is sufficient, no more).
For this reasons, the problem file should be defined in a way that does not introduce any form of inconsistency:
for example, we should not have the situation in which a person initially has a crate containing food and at 
the same time the person needs food.

% We assume that a crate already delivered is no more available to be moved by the robot, even if this assumption
% is not strictly required.


Initially, we need to consider the main object types used in the proposed scenario.
We can simply spot the following types: \texttt{robot}, \texttt{person}, \texttt{crate}, \texttt{location}, 
\texttt{content}, and, from the \textit{Problem 2} on, the assignment introduces the concept of \texttt{carrier}.
% Each type can be easily understood without further explanations. 

We can now start describing the modeled PDDL predicates, actions and functions.
In order to model the location of robotic agents, people and crates, appropriate predicates should be defined.
We could simply use just one predicate to express objects location, but we preferred to define a predicate for 
each type of object to improve clarity. In this way, we have defined the predicates 
\texttt{(robot\_at ?r - robot ?l - location)}, \texttt{(person\_at ?p - person ?l - location)} and 
\texttt{(crate\_at ?c - crate ?l - location)}.



As we have said, people need specific emergency supplies. To model this fact, the predicate
\texttt{(need ?p - person ?co - content)} has been defined. The predicate 
\texttt{(have ?p - person ?co - content)} is used to define that a person has a crate with that content.
In order to know if a crate has already been delivered, the predicate \texttt{(available ?c - crate)}
is used: we assume that when a crate is delivered is no more available to be picked up or moved.
This assumption is not strictly required, even if it sounds reasonable, and could be omitted in the modelling.


To model the fact that a robot can be empty or can hold a crate, the predicates \texttt{(empty ?r - robot)} and
\texttt{(hold ?r - robot ?c - crate)} are respectively used.
We note that the predicate \texttt{empty} is not strictly required but it is useful to simplify the modelling
and to increase the efficiency of the planner.


Three different actions have been modeled: \texttt{pick\_up}, 
\texttt{move} and \texttt{deliver}.
The action \texttt{pick\_up}, as the name suggests, models the robot that picks up a crate:
the robot and the crate should be at the same location, the robot should be empty and the crate available.
The action \texttt{move} indicates the movement of the robot from one location to another.
The action \texttt{deliver} indicates the delivering of a crate to a person: the person should need emergency
supplies of the same type of the content of the crate, the robot should hold the crate and should be at the 
same location of the person.


From \textit{Problem 2} on, the introduction of the carrier, poses the need to change a bit the model 
introducing the \texttt{carrier} type and the \texttt{(carrier\_at ?ca - carrier ?l - location)} predicate.
Instead, the predicates \texttt{hold} and \texttt{empty} have been removed in favour of the predicate
\texttt{(load ?ca - carrier ?c - crate)} which indicates that a crate has been loaded into the carrier.
In order to model the variable capacity of the carriers, the function \texttt{(capacity ?ca - carrier) - number}
has been defined: the function is used to check the remaining capacity of a carrier before loading a crate.
The action \texttt{move\_carrier} has also been added and indicates the robot that moves a specific carrier 
from one location to another.
Instead, the action \texttt{move} indicates now the movement of the robot without bringing a carrier with it.
This action could be useful, for example if we want to model the fact that the carrier is not initially at 
the same location of the robot.

In \textit{Problem 3}, the actions have been transformed into durative actions, specifying appropriate durations.
A constant duration time has been assigned to each action: we decided that \texttt{pick\_up} takes 2 time 
units to execute, \texttt{deliver} takes just 1 time unit, \texttt{move} takes 3 time units and 
\texttt{move\_carrier} takes 4 time units since moving a carrier requires more effort.
A new predicate \texttt{(free ?r - robot)}, that indicates when a robot is not executing any action,
has been added in order to help modelling which actions can be executed in parallel: we decided to model the 
fact that a robot cannot perform more actions in parallel because it sounds more reasonable.
% a robot can perform
% an action only when it is not performing any other action.
% We decided to model the fact that actions performed from the same robot cannot be executed
% in parallel (i.e., the same robot cannot perform the same action ).
% We decided to model the fact that a robot cannot perform the same action or whatever any other action in 
% parallel. 
This does not imply that actions cannot be executed in parallel since more robots can perform
actions at the same time.
% Proper conditions and effects of actions have been implemented in order to enforce this situation, in 
% particular by the use of the predicate \texttt{free}.

In \textit{Problem 4}, we have implemented the \textit{Problem 3} within the \texttt{PlanSys2} framework.
To do this, we have initially created the project workspace and we have copied within it the PDDL domain 
file of \textit{Problem 3}.
Then we have implemented the four action nodes: they can be defined as \textit{fake actions} in the sense 
that they do not really make any useful work but they only show to the user the execution progress of the 
required action. This feedback is given by the function \texttt{do\_work()} that, in
general, should define the code necessary to perform the action.
Finally, we have implemented the launcher, a \texttt{Python} script that is responsible for selecting the
domain and running the action nodes.
% and we have implemented the 
% launcher and the four action nodes. Each action node, inside the function \texttt{do\_work()}, shows
% to the user the progress during the execution of the required action. The action nodes are then instantiated by 
% the launcher.

The default planner used by \texttt{PlanSys2} is \texttt{POPF} and it is also the one used in this project.

