\section{Results}
\label{sec4}
In this section, we present and analyze the obtained results regarding the execution of the planners on 
the PDDL files.

The submitted archive contains the solutions to the presented problems and it is organized as follows.
We have created four folders, one for each problem, that contain the PDDL domain files and one ore more 
PDDL problem files. For each problem file it is reported in a separate file the plan found by the planner.
For Problem 4, there is a folder that contains all the code required to execute the \texttt{PlanSys2} 
problem.
% For each problem there is a folder that contains the PDDL domain file and one ore more PDDL problem files.
% the plan obtained running the planner on each problem file is reported in 


\subsection{Problem 1}
For this problem, we have modeled two PDDL problem files with increasing complexity.
The planner used to found the solution is \texttt{Downward} using the option \texttt{--alias lama-first}.

In the first problem modeled, there are 4 people, 5 crates, 4 locations and the planner successfully
found an optimal solution that requires the execution of 15 actions.

The second problem modeled is a bit more complex: the number of objects modeled increase to 8 people, 
8 crates and 5 locations. Also in this case, the planner successfully
found an optimal solution that requires the execution of 27 actions.


\subsection{Problem 2}
For this problem, the planner used is \texttt{Enhsp-2020} that is contained in \texttt{Planutils}.

The modeled PDDL problem contains 6 people, 6 crates, 4 locations and just one carrier with capacity 4.
Executing the planner with the option \texttt{-anytime}, it managed to find an optimal solution that 
requires performing 16 actions.

\subsection{Problem 3}
For this problem, the planner used is \texttt{Optic} that is contained in \texttt{Planutils}.

It has been decided to model two PDDL problem files.
In the second one, there are two robots and two carriers while in the first one only one of both.
In this way, as previously explained, only in the second problem the actions can be perfomed in parallel.

The planner managed to find a plan of duration 34 seconds for for the first problem and a plan of
duration 24 seconds for the second one.
As can be seen in the second plan, some actions are performed simultaneously as desired.

\subsection{Problem 4}
For this problem, the domain file is the same as Problem 3. TODO
% The \texttt{PlanSys2} framework
The terminal commands required to specify the problem are the equivalent of the PDDL problem file of the 
Problem 3 and are reported in the directory of Problem 4.


\begin{figure}[htbp]
\centerline{\includegraphics[scale=0.45]{assignment_terminal.png}}
\caption{Plan found by \texttt{POPF} planner.}
\label{fig:plan}
\end{figure}
    
\begin{figure}[htbp]
\centerline{\includegraphics[scale=0.365]{assignment_run.png}}
\caption{Execution of the plan.}
\label{fig:execution}
\end{figure}

The \texttt{POPF} planner managed to find the plan of duration 42 seconds shown in Figure \ref{fig:plan}.  
This plan is then executed by \texttt{PlanSys2} within the \texttt{ROS 2} framework. Figure 
\ref{fig:execution} shows the execution of the plan and the messages sended by the action nodes.
