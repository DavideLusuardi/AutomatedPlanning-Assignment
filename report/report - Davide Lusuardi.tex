\documentclass[conference]{IEEEtran}
\IEEEoverridecommandlockouts
% The preceding line is only needed to identify funding in the first footnote. If that is unneeded, please comment it out.
\usepackage{cite}
\usepackage{url}
\usepackage[english]{babel}
\usepackage{amsmath,amssymb,amsfonts}
\usepackage{algorithmic}
\usepackage{graphicx}
\graphicspath{ {./images/} }
\usepackage{textcomp}
\usepackage{xcolor}
\def\BibTeX{{\rm B\kern-.05em{\sc i\kern-.025em b}\kern-.08em
    T\kern-.1667em\lower.7ex\hbox{E}\kern-.125emX}}
\begin{document}

\title{Assignment for the course \\Automated Planning Theory and Practice\\
% {\footnotesize \textsuperscript{*}Note: Sub-titles are not captured in Xplore and
% should not be used}
% \thanks{Identify applicable funding agency here. If none, delete this.}
}

\author{\IEEEauthorblockN{Davide Lusuardi}
\IEEEauthorblockA{\textit{Department of information engineering and computer science} \\
\textit{University of Trento}\\
Trento, Italy \\
davide.lusuardi@studenti.unitn.it}
}

\maketitle

\begin{abstract}
This report is intended to present and analyze the assignment for the course 
\textit{Automated Planning Theory and Practice} \ref{TODO:assignment_document}
and discuss some design choices regarding its modeling and implementation.

The assignment is inspired by an emergency services logistics scenario where
a set of robotic agents have the task to deliver crates containing emergency 
supplies to some injured people.

The assignment is divided into four subproblems, each building on the previous 
one with increasing complexity.
In the first problem, the robot can pick up and move just one crate at time.
In the second problem, the complexity increase a bit given that the robot
exploits a carrier that can load up to four crates.
In the third problem, is required to use durative actions in order to assign
reasonable duration to different actions and model which actions can be executed
in parallel.
Lastly, the third problem has to be implemented within the \texttt{PlanSys2} 
framework, executing in a simulated environment the plan.


% This report is intended to present and discuss some design choices regarding 
% the modeling and implementation of the scenario and the four subproblems 
% proposed in the assignment document \ref{TODO}.
% In this paper, we present and discuss some methods for free-viewpoint synthesis for soccer games.
% These techniques permit to generate novel views of actions from any angle and allow viewers to virtually fly through real soccer scenes.

% In this document, we discuss some methods to accomplish free-viewpoint video visualization for soccer scenes.
% These methods generate novel views of actions from any angle and allow viewers to virtually fly through real soccer scenes.
% TODO: complete
% and is of interest for visualization in TV productions.
\end{abstract}

% \begin{IEEEkeywords}
% component, formatting, style, styling, insert
% \end{IEEEkeywords}

\section{Introduction}
Planning is a branch of Artificial Intelligence that seeks to automate reasoning about
formulating a plan to achieve a given goal in a given situation. 
Planning is a model-based approach: a planning system takes as input a model of the initial state, 
the actions available to change it and the goal conditions, producing as output a plan 
that will reach the goal from the initial situation.

The Planning Domain Definition Language (PDDL) is a formal knowledge representation language 
designed to express planning models. Developed by the planning research community, 
it has become a de-facto standard language of many planning systems.

The purpose of the assignment is two-fold: first, to model the given scenario using the PDDL language 
and find a plan using a state-of-the-art planner; second, leveraging the \texttt{PlanSys2} \cite{PlanSys2} 
infrastructure, to integrate the model within a robotic setting.

ROS2 Planning System, \texttt{PlanSys2} in short, is a framework for symbolic planning that incorporates 
novel approaches for execution on robots working in demanding environments \cite{PlanSys2}.

The proposed scenario is inspired by an emergency services logistic problem:
several injured people are situated at known locations and one or more robotic agents
have the task to deliver crates containing emergency supplies to each person. 
The planning system should formulate a plan for the robotic agents in order to deliver all
the crates needed by the injured people.

The document is structured as follows: in Section \ref{sec2},
we explain in details our understanding of the proposed scenario and the four subproblems; 
in Section \ref{sec3}, we describe the design choices and the proposed solutions to solve the problem,
making some assumptions in order to constrain the problem; % TODO: scrivere meglio
in Section \ref{sec4}, we discuss and analyse the achieved results and briefly describe the
content of the submitted archive; in Section \ref{sec5}, we make some conclusions and future 
works are proposed.




\section{Understanding of the problem}
\label{sec2}

The assignment proposes four subproblems within the same scenario, each building on the previous one.

The proposed scenario consists of an emergency situation in which there are a number of injured people 
who need emergency supplies.
% to help delivering them some crates containing
Each injured person is located in a specific place and may need some kinds of emergency supply
(e.g. food, medicine, beverage, ...).
Each crate is at a specific location and contains just one specific kind of emergency supply.
Some people may already have some crates and some others may not need any crate.
People don't care which crate exactly they get, only the content type is important.
There can be one or more robotic agents that cooperate in order to deliver crates to people.
Each location is connected to every other location, so the robotic agent can move directly between them.
Initially, the robotic agents and the crates are situated at the depot, where there are not injured people.

The above considerations are common for all the four subproblems that we are going to discuss.

\subsection{Problem 1}
This is the easier problem.
We have that a single robotic agent is present at the depot.
It can perform the following actions: pick up and load just one crate if it is at the same location;
move to another location, moving also the loaded crate if present; 
deliver the loaded crate to a person that is at the same location of the robot.

\subsection{Problem 2}
In this subproblem, the robotic agent can move crates in an different way.
We introduce the concept of carrier that can be loaded with up to a specific number of crates. 
This number is problem specific and is modeled in the problem file.
The robotic agent loads the carrier and moves it. 
At the location of the robot, it can load some crates on the carrier or deliver some others to people.

The difficult part consists in modeling correctly the carrier and its variable capacity.

\subsection{Problem 3}
In this subproblem, it's required to make use of durative actions
% , assigning them reasonable durations 
% and the possibility to execute them in parallel.
Reasonable durations should be assigned to actions coherently with their time spans.
The possibility to execute actions in parallel should be analyzed taking into account real constraints.

\subsection{Problem 4}
Problem 3 has been integrated within the PlanSys2 infrastructure.
% TODO: spiegare meglio



\section{Design Choices}
\label{sec3}

In this section, we discuss some design choices and some assumptions that have been made 
analyzing the scenario and the different subproblems. These assumptions are required 
in order to better delineate the problem.
% during the implementation of the PDDL domain and problem files.

We can start saying that each crate can have just one specific content, each person can need none or 
more emergency supplies and can initially have some crates. This implies that each person could need 
more crates.
% Each person , can need some emergency supplies of specific content, or can need nothing.
We assume that a single crate with specific content is enough to satisfy the person need of that emergency 
supply, i.e. if a person needs some food, it is sufficient to deliver it just one crate containing food, no more. 
% We assume that the need of person of a specific content can be satisfied through the delivering 
% to it of just one crate having that content (e.g. if Mario needs food, just one crate containing
% food is sufficient, no more).
For this reasons, the problem file should be defined in a way that does not introduce any form of inconsistency:
for example, we should not have the situation in which a person initially has a crate containing food and at 
the same time the person needs food.

% We assume that a crate already delivered is no more available to be moved by the robot, even if this assumption
% is not strictly required.


Initially, we need to consider the main object types used in the proposed scenario.
We can simply spot the following types: \texttt{robot}, \texttt{person}, \texttt{crate}, \texttt{location}, 
\texttt{content}, and, from the \textit{Problem 2} on, the assignment introduces the concept of \texttt{carrier}.
% Each type can be easily understood without further explanations. 

We can now start describing the modeled PDDL predicates, actions and functions.
In order to model the location of robotic agents, people and crates, appropriate predicates should be defined.
We could simply use just one predicate to express objects location, but we preferred to define a predicate for 
each type of object to improve clarity. In this way, we have defined the predicates 
\texttt{(robot\_at ?r - robot ?l - location)}, \texttt{(person\_at ?p - person ?l - location)} and 
\texttt{(crate\_at ?c - crate ?l - location)}.



As we have said, people need specific emergency supplies. To model this fact, the predicate
\texttt{(need ?p - person ?co - content)} has been defined. The predicate 
\texttt{(have ?p - person ?co - content)} is used to define that a person has a crate with that content.
In order to know if a crate has already been delivered, the predicate \texttt{(available ?c - crate)}
is used: we assume that when a crate is delivered is no more available to be picked up or moved.
This assumption is not strictly required, even if it sounds reasonable, and could be omitted in the modelling.


To model the fact that a robot can be empty or can hold a crate, the predicates \texttt{(empty ?r - robot)} and
\texttt{(hold ?r - robot ?c - crate)} are respectively used.
We note that the predicate \texttt{empty} is not strictly required but it is useful to simplify the modelling
and to increase the efficiency of the planner.


Three different actions have been modeled: \texttt{pick\_up}, 
\texttt{move} and \texttt{deliver}.
The action \texttt{pick\_up}, as the name suggests, models the robot that picks up a crate:
the robot and the crate should be at the same location, the robot should be empty and the crate available.
The action \texttt{move} indicates the movement of the robot from one location to another.
The action \texttt{deliver} indicates the delivering of a crate to a person: the person should need emergency
supplies of the same type of the content of the crate, the robot should hold the crate and should be at the 
same location of the person.


From \textit{Problem 2} on, the introduction of the carrier, poses the need to change a bit the model 
introducing the \texttt{carrier} type and the \texttt{(carrier\_at ?ca - carrier ?l - location)} predicate.
Instead, the predicates \texttt{hold} and \texttt{empty} have been removed in favour of the predicate
\texttt{(load ?ca - carrier ?c - crate)} which indicates that a crate has been loaded into the carrier.
In order to model the variable capacity of the carriers, the function \texttt{(capacity ?ca - carrier) - number}
has been defined: the function is used to check the remaining capacity of a carrier before loading a crate.
The action \texttt{move\_carrier} has also been added and indicates the robot that moves a specific carrier 
from one location to another.
Instead, the action \texttt{move} indicates now the movement of the robot without bringing a carrier with it.
This action could be useful, for example if we want to model the fact that the carrier is not initially at 
the same location of the robot.

In \textit{Problem 3}, the actions have been transformed into durative actions, specifying appropriate durations.
A constant duration time has been assigned to each action: we decided that \texttt{pick\_up} takes 2 time 
units to execute, \texttt{deliver} takes just 1 time unit, \texttt{move} takes 3 time units and 
\texttt{move\_carrier} takes 4 time units since moving a carrier requires more effort.
A new predicate \texttt{(free ?r - robot)}, that indicates when a robot is not executing any action,
has been added in order to help modelling which actions can be executed in parallel: we decided to model the 
fact that a robot cannot perform more actions in parallel because it sounds more reasonable.
% a robot can perform
% an action only when it is not performing any other action.
% We decided to model the fact that actions performed from the same robot cannot be executed
% in parallel (i.e., the same robot cannot perform the same action ).
% We decided to model the fact that a robot cannot perform the same action or whatever any other action in 
% parallel. 
This does not imply that actions cannot be executed in parallel since more robots can perform
actions at the same time.
% Proper conditions and effects of actions have been implemented in order to enforce this situation, in 
% particular by the use of the predicate \texttt{free}.

In \textit{Problem 4}, we have implemented the \textit{Problem 3} within the \texttt{PlanSys2} framework.
To do this, we have initially created the project workspace and we have copied within it the PDDL domain 
file of \textit{Problem 3}.
Then we have implemented the four action nodes: they can be defined as \textit{fake actions} in the sense 
that they do not really make any useful work but they only show to the user the execution progress of the 
required action. This feedback is given by the function \texttt{do\_work()} that, in
general, should define the code necessary to perform the action.
Finally, we have implemented the launcher, a \texttt{Python} script that is responsible for selecting the
domain and running the action nodes.
% and we have implemented the 
% launcher and the four action nodes. Each action node, inside the function \texttt{do\_work()}, shows
% to the user the progress during the execution of the required action. The action nodes are then instantiated by 
% the launcher.

The default planner used by \texttt{PlanSys2} is \texttt{POPF} and it is also the one used in this project.


\section{Results}
\label{sec4}
In this section, we present and analyze the obtained results regarding the execution of the planners on 
the PDDL files.

The submitted archive contains the solution to the presented problems and it is organized as follows.
We have created four folders, one for each problem, that contain the PDDL domain file and one ore more 
PDDL problem files. For each problem file it is reported in a separate file the plan found by the planner.
% For each problem there is a folder that contains the PDDL domain file and one ore more PDDL problem files.
% the plan obtained running the planner on each problem file is reported in 


\subsection{Problem1}
For this problem, we have modeled two PDDL problem files with increasing complexity.
The planner used to found the solution is \texttt{Downward} using the option \texttt{--alias lama-first}.

In the first problem modeled, there are 4 people, 5 crates, 4 locations and the planner successfully
found an optimal solution that requires the execution of 15 actions.

The second problem modeled is a bit more complex: the number of objects modeled increase to 8 people, 
8 crates and 5 locations. Also in this case, the planner successfully
found an optimal solution that requires the execution of 27 actions.


\subsection{Problem2}
For this problem, the planner used is \texttt{Enhsp-2020} that is contained in \texttt{Planutils}.

The modeled PDDL problem contains 6 people, 6 crates, 4 locations and just one carrier with capacity 4.
Executing the planner with option \texttt{-anytime}, it managed to find an optimal solution that 
requires performing 16 actions.

\subsection{Problem3}
For this problem, the planner used is \texttt{Optic} that is contained in \texttt{Planutils}.

It has been decided to model two PDDL problem files.
In the second one, there are two robots and two carriers while in the first one only one of both.
In this way, as previously explained, only in the second problem the actions can be perfomed in parallel.

The planner managed to find a plan of duration 34 seconds for for the first problem and a plan of
duration 24 seconds for the second one.
As can be seen in the second plan, some actions are performed simultaneously as desirable.

\section{Conclusions}
\label{sec5}

We have shown how a typical planning problem like the emergency services logistics scenario 
proposed in this report can be solved using the PDDL language and state-of-the-art planners.

Problems with increasing complexity have been discussed and solved, showing how modern planners can 
efficiently found complex plans.
% This remarks the importance of automated planning for many real applications.

Lastly, we have shown how to integrate the planning problem within the \texttt{PlanSys2} infrastructure
in order to be able to execute the plan in a simulated environment.

In future works, using real robotic agents, it could be interesting to implement real action nodes
and to study the problems that arise in a real environment.

% In future works, it could be interesting use real robotic agents and implement real action nodes in order 
% to solve 

It could be also interesting analyse the proposed problems from a hierarchical point of view, extending the 
PDDL domain and problem files using the Hierarchical Domain Description Language \cite{hddl} and studying 
the hierarchical plans obtained.



\input{old1.tex}
\input{old2.tex}
\input{old3.tex}
\input{old4.tex}
\input{old5.tex}

\input{sec06_conclusion.tex}



\bibliographystyle{IEEEtran}
\bibliography{biblio.bib}



\end{document}